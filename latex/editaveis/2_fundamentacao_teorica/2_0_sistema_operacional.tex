\section{Sistemas Operacionais}
    
    Um sistema operacional pode ser considerado um grande software que faz a interface entre o usuário e os componentes de hardware do computador permitindo, por exemplo, a disponibilização de uma interface gráfica. Seu objetivo é gerenciar o compartilhamento de recursos do sistema.
      
    Sistemas operacionais atuais permitem o armazenamento prolongado de dados, a existência e a utilização de vários usuários simultaneamente. Dessa forma, eles são responsáveis por gerenciar o uso dos recursos disponíveis no computador, resolver eventuais conflitos de requisições simultâneas, controlar o acesso aos dados armazenados, dentre outras tarefas.
    %https://dl.acm.org/doi/pdf/10.5555/540365

    \subsection{FreeBSD}

        O FreeBSD é um sistema operacional de código aberto derivado do BSD (Berkeley Software Distribution), versão do UNIX desenvolvido pela Universidade da Califórnia em Berkeley. Ele é considerado um sistema operacional completo, o que quer dizer que o sistema entrega kernel, drivers, espaço de usuário e documentação. É utilizado por grandes empresas ao redor do mundo pois o código fonte do freeBSD é entregue sob uma licença BSD permissiva, a qual permite que qualquer pessoa possa fazer modificações e não tenha que disponibilizar publicamente essas alterações, o que torna muito mais atrativa para empresas privadas. %site do freeBSD

        É válido ressaltar que toda a pilha de protocolos TCP/IP da ARPANET foi implementada utilizando o BSD, que foi essencial para a criação e existência da internet. %https://www.ime.usp.br/~is/abc/abc/node23.html
