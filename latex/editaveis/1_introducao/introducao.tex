\chapter[Introdução]{Introdução}
%\addcontentsline{toc}{chapter}{Introdução}




%Não resta dúvida de que atualmente o trânsito seria inviável sem a presença de normas, de regras que regulam a movimentação e a ocupação do espaço viário. Toda a supervisão do trânsito (fiscalização, controle) está baseada num conjunto de leis e dispositivos, que constam nos códigos e regulamentos de cada país. Em toda a experiência mundial de trânsito, as normas juntamente com a fiscalização do seu cumprimento têm exercido um papel fundamental, na medida em que é muito notável a relação direta entre fiscalização-punição e comportamento adequado (não infração às normas) com a diminuição de acidentes, e inversamente entre a impunidade e o comportamento inadequado (infração às normas) com aumento de acidentes. Um sistema normativo é condição indispensável para a fluidez e segurança no trânsito, sendo esta, sempre relacionada (e medida) aos índices de acidentes \cite{article_transito_alcool}.
%
%Em 2018 foram registradas pelas autoridades de trânsito somente no Distrito Federal um total 2.740.685 infrações de trânsito, o que corresponde a uma média diária de aproximadamente 7500 infrações. Este número se torna ainda maior quando considerado os registros realizados por órgãos de outras unidades federativas, e também pelos órgãos responsáveis por fiscalizações em âmbito federal como a Polícia Rodoviária Federal.
%
%A forma e o sistema em que o registro da infração de trânsito é feito depende de fatores como o órgão autuador e a localização geográfica da mesma. Esse forma de registro não padronizada, centralizada pelas autoridades e condicionado a diversos fatores fazem com que os serviços associados aos órgãos de trânsito, como aplicação de multas ou realização de possíveis recursos, se tornem ineficientes e lentos, o que eleva o custo de manutenção para o Estado e aumenta a insatisfação de todos os cidadãos que precisam utilizar algum destes serviços.
%
%De forma a integrar esta capilaridade de sistemas presentes nas Unidades Federativas, o sistema RENAINF (Registro Nacional de Infrações de Trânsito) foi criado e é atualmente coordenado pelo DENATRAN (Departamento Nacional de Trânsito). Porém este sistema somente é utilizado para registrar as infrações de trânsito cometidas em unidade federada diferente daquela de onde o veículo estiver registrado e licenciado, bem como para o registro das infrações impostas pelas autoridades de trânsito federadas independente da vinculação de registro do veículo. Este sistema possibilita que o órgão autuador obtenha os dados necessários para registrar a informação da infração cometida e vincular estes débitos no Departamento Estadual de Trânsito  (DETRAN) de registro do veículo, não sendo assim um sistema realmente unificado para o registro de infrações de trânsito no território brasileiro \cite{renainf_fazenda_sp}.
%
%A construção de um sistema nacional único e centralizado por uma autoridade como o DENATRAN poderia mitigar problemas decorrentes dessa capilaridade de sistemas, porém esta centralização acaba tendo seus pontos negativos como um maior custo de manutenção e maior dificuldade para garantir a integridade do sistema. Em meados de 2009 uma nova tecnologia denominada Blockchain foi proposta para o uso em uma aplicação monetária que não necessitava de uma unidade central de confiança para seu funcionamento, e logo observou-se que seu conceito poderia ser empregado em contextos diferentes do que foi idealizado por possibilitar algumas vantagens como a eliminação de intermediários e o aumento da segurança com custo baixo \cite{beginnig_blockchain_bikramaditya}.


\section{Justificativa}

Alunos que cursam engenharia de software devem aprender sobre diferentes áreas para que possam planejar o desenvolvimento de um software. Nos dias atuais, raros são os softwares que não estão conectados a uma rede de computadores, seja a internet ou uma rede privada. Portanto, entender o funcionamento de uma rede de computadores se torna essencial para que esse futuro profissional aprenda a configurar uma rede, resolva eventuais problemas e não fique dependente de um profissional da área de redes. Um administrador de sistemas não precisa conhecer a fundo os conceitos de redes, mas saber o suficiente para diagnosticar seus próprios problemas transforma um bom administrador de sistema em um ótimo \cite{Lucas2019}.

É extremamente importante que um profissional da área de tecnologia expanda sua gama de conhecimento para lidar com diferentes tecnologias durante sua carreira. O FreeBSD e o Linux são sistemas operacionais que funcionam muito bem para a configuração e administração de uma rede de computadores, os dois são completamente distintos, porém ambos são baseados no UNIX. Por esse motivo eles possuem algumas semelhanças que fazem a curva de aprendizado ficar menor quando se trata de migração de um sistema para o outro, mas ao mesmo tempo introduz novos conceitos e uma nova bagagem de aprendizado.

O modelo tradicional de ensino propõe que o professor seja o detentor do conhecimento passando o conteúdo enquanto os alunos absorvem tudo de maneira passiva. No ensino por meio da pratica, que faz parte da metodologia ativa de aprendizagem, os alunos são estimulados a tomarem a frente, com maior interação e independência, participando ativamente do processo, fazendo com que o conhecimento seja realmente absorvido (https://sanare.emnuvens.com.br/sanare/article/view/1049/595). %\cite{beginnig_blockchain_bikramaditya}

Sendo assim, o aprendizado de conceitos e fundamentos de uma rede de computadores fica mais fácil de entender quando o aluno se torna parte desse processo por meio da prática. Portanto, este trabalho tem como proposta apresentar um conjunto de experimentos práticos para operação básica de uma rede de computadores.


\section{Objetivos}

    \subsection{Objetivo Geral}
    
    O objetivo principal da pesquisa é desenvolver um conjunto de experimentos práticos para operação básica de uma rede de computadores, utilizando o FreeBSD, para ser aplicado na disciplina “fundamentos de redes de computadores”, da Faculdade do Gama na Universidade de Brasília, que atualmente é ensinado utilizando o Linux.
    
    \subsection{Objetivos Específicos}
    
    %\begin{enumerate}
    %    \item Desenvolver uma aplicação descentralizada acessível a todos que desejarem;
    %    \item Estudar sobre a nova tecnologia \textit{blockchain} e como ela pode ser usada em diversas áreas;
    %    \item Demonstrar a viabilidade do uso de aplicações descentralizadas em um contexto atualmente centralizado;
    %\end{enumerate}
    
    
\section{Estrutura do Documento}

Este documento está composto pelos seguintes capítulos:

\begin{enumerate}
    \item \textbf{Introdução:} A introdução contém um breve contexto e motivações para a realização deste trabalho;
    \item \textbf{Fundamentação Teórica:} A fundamentação teórica contém conceitos teóricos considerados necessários para uma melhor compreensão do trabalho que será desenvolvido;
    \item \textbf{Proposta de Trabalho:} Contém uma explanação da proposta de trabalho à ser desenvolvido;
    \item \textbf{Metodologia:} Contém descrição da forma que o projeto será desenvolvido, assim como metodologias, políticas e ferramentas que serão utilizadas.
    \item \textbf{Resultados e Discussões:} Contém uma explanação da solução desenvolvida, contendo informações relacionadas as ferramentas utilizadas, arquitetura e considerações.
    \item \textbf{Conclusão:} Contém as considerações finais a cerca do trabalho desenvolvido
\end{enumerate}
