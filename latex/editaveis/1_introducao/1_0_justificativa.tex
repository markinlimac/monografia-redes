\section{Justificativa}

Alunos que cursam bacharelado em Engenharia de Software devem aprender sobre diferentes áreas para que possam planejar o desenvolvimento de um software. Nos dias atuais, raros são os softwares que não estão conectados a uma rede de computadores, seja a internet ou uma rede privada. Portanto, entender o funcionamento de uma rede de computadores se torna essencial para que esse futuro profissional aprenda a configurar uma rede, resolva eventuais problemas e não fique dependente de um profissional da área de redes. Um administrador de sistemas não precisa conhecer a fundo os conceitos de redes, mas saber o suficiente para diagnosticar seus próprios problemas transforma um bom administrador de sistema em um ótimo \cite{Lucas2019}.

É extremamente importante que um profissional da área de tecnologia expanda sua gama de conhecimento para lidar com diferentes tecnologias durante sua carreira. O FreeBSD e o Linux são sistemas operacionais que funcionam muito bem para a configuração e administração de uma rede de computadores e, embora ambos sejam baseados no UNIX, cada um apresenta características que permitem clara distinção. Por esse motivo eles possuem algumas semelhanças que fazem a curva de aprendizado ficar menor quando se trata de migração de um sistema para o outro, mas ao mesmo tempo introduz novos conceitos e uma nova bagagem de aprendizado.

O modelo tradicional de ensino propõe que o professor seja o detentor do conhecimento passando o conteúdo enquanto os alunos absorvem tudo de maneira passiva. No ensino por meio da prática, que faz parte da metodologia ativa de aprendizagem, os alunos são estimulados a tomarem a frente, com maior interação e independência, participando ativamente do processo, fazendo com que o conhecimento seja realmente absorvido \cite{MetodologiaAtiva}.%https://sanare.emnuvens.com.br/sanare/article/view/1049

Sendo assim, o aprendizado de conceitos e fundamentos de uma rede de computadores fica mais fácil quando o aluno se torna parte desse processo por meio da prática. Portanto, este trabalho tem como proposta apresentar um conjunto de experimentos práticos para exercitar a operação básica de uma rede de computadores.